\documentclass[a4paper,12pt, oneside]{book}
\usepackage[noheadfoot, nomarginpar, top=1.7cm, bottom=1.7cm, left=1.5cm, right=1.5cm]{geometry}
\usepackage[T1]{fontenc}
\usepackage[utf8]{inputenc}
\usepackage{lmodern}%
\usepackage[french]{babel}
\usepackage{graphicx}
\usepackage{pstricks} %Couleurs
\newcommand{\titreRapport}{Refonte du système multilingue pour une application web}
\newcommand{\mbx}{Memobox~/\-~2G Technologies}
\newcommand{\memobox}{Memobox}
\newcommand{\techno}{2G Technologies}
\newcommand{\adt}{\bsc{AUDITEL}com}

\newcommand{\Adr}{Antoine de \bsc{Roquemaurel}}
\newcommand{\adr}{A. de \bsc{Roquemaurel}}
\newcommand{\Romain}{Romain \bsc{Auriac}}
\newcommand{\Denis}{Denis \bsc{Mallet}}
\newcommand{\romain}{R. \bsc{Auriac}}
\newcommand{\denis}{D. \bsc{Mallet}}
\newcommand{\mlanguage}{\texttt{MemoLanguage}}

\pagestyle{empty}

\definecolor{grisgris}{gray}{0.4}
\definecolor{bleu}{rgb}{0.02, 0.28, 0.5}
\fboxsep =0pt \parindent =0pt\parskip =12pt
\begin{document}
	\sffamily
	% Le cadre sur la gauche, une boite rouge avec les logos IUT et UPS
	\fcolorbox{bleu}{bleu}{
		\parbox[t] [0.991\textheight][b]{0.265\textwidth}{%
			\begin{center}
			\vspace{10px}
			\includegraphics[width=4.5cm]{images/ups.jpg}\\%
			\vspace{5px}
			\includegraphics[width=4.5cm]{images/logoIUT.png}\\%
			\vfill
			\includegraphics[width=4.5cm]{images/logoMemobox.png}\\%
			\fontsize{11.5}{11.5}\selectfont \color{white}{
			Memobox -- 2G Technologies\\
			12 Bv de l'Europe\\
			31850 -- \bsc{Montrabé}
			}
			\end{center}
			\vspace{10px}
		}
	}
	%
	\hspace{0.05\textwidth} %une petite marge sur la droite du cadre
	%
	% Pour éviter de créer un nouveau paragraph, on commente les lignes vide
	\parbox[t] [0.991\textheight]{0.66\textwidth}{\large%
		\vspace{30px}
		%
		\mbox{}\hfill{
			\begin{flushright}
				\fontsize{15}{15} \selectfont \Adr{}\\\vspace{10px}
				\fontsize{10}{10} \selectfont Stage effectué chez Memobox -- 2G technologies\\
				Du 10 Avril 2012 au 22 Juin 2012
			\end{flushright}
			\begin{flushleft}
				Pour M. Denis \bsc{Mallet}\\
				Pour M. Patrick \bsc{Magnaud}\\
			\end{flushleft}
		}\par
		%
		\vfill\mbox{}\vfill
		%
		%
		\fbox{
		\begin{minipage}{0.66\textwidth}
			\begin{center}
				\vspace{20px}	
				\fontsize{35}{25}\selectfont Rapport de stage
				
				\fontsize{22}{22}\selectfont \titreRapport{}
				\vspace{20px}	
			\end{center}
		\end{minipage}
		}
		%
%		\hrule height 4pt
		%
		\begin{center}
			\fontsize{20}{20}\selectfont\vspace{20px} --- Annexes ---
		\end{center}
		% Les vfill servent à insérer du blanc en dessous du titre pour éviter que le titre aille tout en bas
		% Plus il y a de vFill plus le titre sera haut dans la page
		\vfill
		\vfill
		\vfill
		\vfill
		\vfill
		\vfill
		\vfill
		\vfill
		\vfill
		\vfill
		\vfill
		\mbox{}
	}%
\end{document}
